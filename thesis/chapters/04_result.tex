In this chapter we describe OpenLogistics, the created decentralized  marketplace prototype for facilitating trustless transport contracts. OpenLogistics uses the bitcoin mainet for the multisignature trustless intermediation mechanism, the distributed data storage bigchaindb to store legally binding liability Ricardian contracts and the software client aggregates the contracts and facilitates the interaction. \par
To analyze the obtained result of OpenLogistics will be devided into three parts representing the three roles which can interact with the mechanism. Every role owns a Bitcoin and BigchainDB keypair once interacting with the software client. The roles which interact with the mechanism are the following:
\begin{enumerate}
  \item Service consumer
  \item Endpoint actor
  \item Service provider
\end{enumerate}
The \textbf{service consumer} uses OpenLogistics because he wants to transport a physical good. He can create and sign a Ricardian contract which will than be stored on BigchainDB. This represents a request for transport, the ownership of the physical good and includes the information regarding the transport, a example of the data model can be found in appendix C. \par
The \textbf{service provider} can acces the OpenLogistics marketplace orderbook and accept the contract in his local client. Upon accepting

What is really being exchanged is the right to the transaction id by being possible to acces it with the private key. What the transaction id represent in the orderbook is proof that asset ownership is possible to acces. The exchange of owner innitiates the escrow balance which is equavalent to the asset value.

The information regarding the physical asset is captured inside an ricardian contract \cite{}, an example of the data model can be found in appendix C. This ricardian contract is stored with bigchaindb and the user which creates and signs the contract owns the contract. This user is the only one who can transfer it to another bigchaindb keypair. This creates a mechanism where verification can always take place because there is full transparancy on assets handled. Ownership of the asset is proofable. A reputation system could find it's place if aggregation of transactions on a specific bigchaindb keypair represents transport contracts completed.

\subsection{Transparancy}

A example of a mainnet testrun can be seen locally, if copy of blockchain is present, or through commonly used block explorers \href{https://www.blocktrail.com/BTC}{[add as footnote]}
 with the following addresses:
\begin{enumerate}
  \item Service consumer: 17F4ZhEJp83qqEG1S6z8YcPbWW7AdqbkZ3
  \item Endpoint actor: 19exDB5Fb2gQAv7k2dH93WbLga1ZUNz9mh
  \item Service provider: 1EY38FGwuSg3uRzetBwYqYh9jjbX55fHsL
  \item Multisig (B+C): 3MiFyavsRpMZBzfxFk94WdeZUnbQP1hdDy
\end{enumerate}

Actor B actually has no UTXO on his address, this is due only using his keypair to sign the transactions. The function Actor B actually fulfills is being the oracle for conflict resolution upon dropoff exchange by signing or not.\par
The asset transferring taking place can also be seen by exploring bigchaindb addresses. \par
The aggregation of the movement of assets between addresses can be used for a reputation system.

% todo: add ricardian contract example of data model in appendix
\subsection{metrics}

Trustless (Conflict intermediation)
Time (blocktime, zero confirmations, other networks)
Privacy (what is known)
Cost (3x transaction fee)
Attack vectors (in conclusion)
