In this chapter we describe OpenLogistics, the created decentralized  marketplace prototype for transport contracts. OpenLogistics uses the bitcoin mainet for the multisignature mechanism and bigchaindb to create orders and aggregate them in a orderbook. \par

To analyze the results the whole testsetup will be devided into different phases, at every phase the possible actor actions will be evaluated. The phases to be analyzed are the following:
\begin{enumerate}
  \item Setup
  \item Pickup
  \item Dropoff
  \item Redeeming
\end{enumerate}

What is really being exchanged is the right to the transaction id by being possible to acces it with the private key. What the transaction id represent in the orderbook is proof that asset ownership is possible to acces. The exchange of owner innitiates the escrow balance which is equavalent to the asset value.

\subsection{Transparancy}

A example of a mainnet testrun can be seen locally, if copy of blockchain is present, or through commonly used block explorers \href{https://www.blocktrail.com/BTC}{[add as footnote]}
 with the following addresses:
\begin{enumerate}
  \item Actor A: 17F4ZhEJp83qqEG1S6z8YcPbWW7AdqbkZ3
  \item Actor B: 19exDB5Fb2gQAv7k2dH93WbLga1ZUNz9mh
  \item Actor C: 1EY38FGwuSg3uRzetBwYqYh9jjbX55fHsL
  \item Multisig (B+C): 3MiFyavsRpMZBzfxFk94WdeZUnbQP1hdDy
\end{enumerate}

Actor B actually has no UTXO on his address, this is due only using his keypair to sign the transactions. The function Actor B actually fulfills is being the oracle for conflict resolution upon dropoff exchange by signing or not.\par
The asset transferring taking place can also be seen by exploring bigchaindb addresses. \par
The aggregation of the movement of assets between addresses can be used for a reputation system.
