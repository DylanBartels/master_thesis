In this chapter we describe OpenLogistics\footnote{\url{https://www.github.com/DylanBartels/master_thesis/src}}, the created decentralized marketplace proof of concept for facilitating trustless transport contracts. OpenLogistics uses the Bitcoin main network for the multisig trustless intermediation mechanism, the distributed data storage bigchaindb to store legally binding liability Ricardian contracts and the software client aggregates the contracts and facilitates the interaction. \par
To analyze the obtained result of OpenLogistics three roles which can interact with the mechanism will be defined. Every role owns a Bitcoin and BigchainDB keypair once interacting with the software client. The roles which interact with the mechanism are the following:
\begin{itemize}
  \item Service consumer
  \item Endpoint actor
  \item Service provider
\end{itemize}
The \textbf{service consumer} uses OpenLogistics because he wants to transport a physical object. He can create and sign a Ricardian contract which will than be stored on BigchainDB. This represents a request for transport, the ownership of the physical object and includes the information regarding the transport, an example of the data model can be found in appendix B. \par

The \textbf{service provider} can acces the OpenLogistics marketplace order-book and accept the contract in his local client. Upon accepting he creates a transaction script and travels to the location of the \textbf{service consumer}. Upon the \textbf{service provider} arriving at the \textbf{service consumer} the \textbf{service consumer} accesses the client and selects his own Ricardian Contract and the pickup exchange takes place illustrated in figure \ref{fig:2 first exchange}. The \textbf{service provider} now moves to the \textbf{endpoint actor} and the \textbf{endpoint actor} accesses the client and the drop-off exchange takes place illustrated in Figure \ref{fig:3 drop-off exchange}. Eventually the \textbf{service provider} endup with $tx_3$ and can claim the reward in his own client for his provided labor. \par

\subsection{Transparency}

Due to the inherit transparancy of the Blockchain it is possible to see the interaction between OpenLogistics and the network. A example testrun of the mechanism can be seen locally, if a Blockchain copy is present, or through commonly used block explorers\footnote{\url{https://www.blocktrail.com/BTC}}
 by searching on the addresses found in Table \ref{tab:addresses}.

\begin{table}[h]
  \begin{tabular}{ll}
    \hline
    \multicolumn{1}{|l|}{Role}                  & \multicolumn{1}{l|}{Address}                            \\ \hline
    \multicolumn{1}{|l|}{Service Consumer}      & \multicolumn{1}{l|}{17F4ZhEJp83qqEG1S6z8YcPbWW7AdqbkZ3} \\ \hline
    \multicolumn{1}{|l|}{Endpoint Actor}        & \multicolumn{1}{l|}{19exDB5Fb2gQAv7k2dH93WbLga1ZUNz9mh} \\ \hline
    \multicolumn{1}{|l|}{Service Provider}      & \multicolumn{1}{l|}{1EY38FGwuSg3uRzetBwYqYh9jjbX55fHsL} \\ \hline
    \multicolumn{1}{|l|}{Multisig (Endpoint Actor, Service Provider)}      & \multicolumn{1}{l|}{3MiFyavsRpMZBzfxFk94WdeZUnbQP1hdDy } \\ \hline
  \end{tabular}
  \caption{Bitcoin addresses used to execute a testun of the mechanism in OpenLogistics}
  \label{tab:addresses}
\end{table}

The \textbf{service provider} actually has no transaction being broadcasted from his address, this is due only using his keypair to sign the transactions from the multisig address. The function \textbf{service provider} actually fulfils is being the oracle for conflict resolution upon drop-off exchange by signing or not.\par
The transferring of Ricardian Contract ownership can be seen by exploring bigchaindb addresses. The aggregation of the ownership change between addresses can be used for a proof of delivery. This proof could be the foundation for a reputation system with further research.

\subsection{Advantages \& Disadvantages}

The advantage of the facilitated mechanism of OpenLogistics is that if all actors agree upon the set rules in the Ricardian Contract \textit{anybody} can participate. We chose to implement the payment and escrow mechanism on the Bitcoin network, which is in the blockchain domain perceived as simpler and less vulnerable in terms of incentive structure than other networks. \par
The downside of the Bitcoin network is that the average blocktime is 10 minutes, which is not consistant which can create long waiting periods when the pickup exchange is taking place. An anomly example of a 1 hour blocktime can be found in appendix C. Other blockchain alternatives can be chosen depending as long as they support multisignature. Another disadvantage of the structure is that you have to spend three transaction fees to make the mechanism work, price fluctuations of the Bitcoin price can disrubt the mechanism.\par
We chose to implement the OpenLogistics client with Javascript libraries which can easily be developed towards mobile. We assumed with the proof of concept that the actor keypair is bounded to a physical location. The signing of the transactions takes place in the actors local client resulting in the actor his private key never leaving the software client bounded physical location.

\subsection{Claim}

We claim that OpenLogistics facilitates peer-to-peer decentralized transport contracts while aiming to create trustless intermediation. Through the evaluation of the research questions in Chapter 5 we will provided evidence of our claim.
