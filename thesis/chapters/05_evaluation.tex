\subsection{Research Questions \& Answers}

In Chapter 1 we stated our research questions and so far we answered them considering the motivating prototype. Generally, we answered the research questions as follows:
\bigbreak
\noindent \textbf{How can you provide a mechanism to facilitate peer-to-peer decentralized trustless transport contracts?} From our research we conclude that to facilitate this mechanism a Bitcoin multisignature escrow incentive strucure can incentivize the transportation without relying on trusted intermediaries.

\bigbreak
\noindent \textbf{Can trustless intermediation exists on this marketplace without a custodian for dispute prevention and resolution?} Trustless intermediation cannot take place with the transport of physical goods. During the process of transport a person will \textit{always} be custodian of the physical good.

\bigbreak
\noindent \textbf{How decentralized is the mechanism?} We claim that the mechanism is as decentralized as the two networks it is build upon.

\bigbreak
\noindent \textbf{To what degree do oracles play a role in the verification of the information?} The verification of information by an oracle takes place twice during the mechanism. The first time by the service provider when he verifies that the physical asset is simalair to the stated Ricardian Contract when it is being picked up. The second time by the endpoint actor when he verifies that the physical asset being delivered is correct.

\bigbreak
\noindent \textbf{What level of anonymity is possible?} We claim that the starting point and endingpoint of the transport identity will always be known. However the transport actor can maintain privacy in this mechanism, the only aspect of privacy he will have to turn in is the sight of his physical apperance to the service consumer and endpoint actor.

\bigbreak
\noindent \textbf{Which attack vectors are possible to undermine this mechanism?} The mechanism uses the PoW solution twice to counteract the double spending possibilities of the escrow and the Ricardian contract proof of ownership. We claim that if the actors behave rational the incentive structure holds.

\subsection{Evidence}
