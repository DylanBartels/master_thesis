\subsection{Research Questions \& Answers}

In Chapter 1 we stated our research questions and so far we answered them considering the motivating prototype. Generally, we answered the research questions as follows:
\bigbreak
\noindent \textbf{How can you provide a mechanism to facilitate peer-to-peer decentralized trustless transport contracts?} From our research we conclude that to facilitate this mechanism a Bitcoin multisignature escrow incentive structure can incentivise the transportation without relying on trusted intermediaries.

\bigbreak
\noindent \textbf{Can trustless intermediation exists on this marketplace without centralized dispute resolution?} Trustless intermediation cannot take place with the transport of physical goods. During the process of transport a person will \textit{always} be custodian of the physical good. Due to this restriction you can never guarantee the expected outcome, centralized conflict resolution will remain to play a roll in transport.
% The provided incentive mechanism assumes that the all actors behave rational during the process and do not deliberately punish themself by working against the mechanism.

% \bigbreak
% \noindent \textbf{How decentralized is the mechanism?} We claim that the mechanism supporting network is as decentralized as the two networks it is build upon.

\bigbreak
\noindent \textbf{To what degree do oracles play a role in the verification of the information?} The verification of information by an oracle takes place twice during the mechanism. The first time by the service provider when he verifies that the physical object is similar to the stated Ricardian Contract when it is being picked up. The second time by the endpoint actor when he verifies that the physical object being delivered is correct.

\bigbreak
\noindent \textbf{What level of anonymity is possible?} We claim that the starting point and endingpoint of the transport identity will always be known. However the service provider can maintain privacy in this mechanism, the only aspect of privacy he will have to turn in is the sight of his physical appearance to the service consumer and endpoint actor.

\bigbreak
\noindent \textbf{Which attack vectors are possible to undermine this mechanism?} The mechanism uses the PoW solution twice to counteract the double spending possibilities of the escrow and the Ricardian contract proof of ownership. We claim that if the actors behave rational the incentive structure holds.
\newpage

\subsection{Evidence}

To outline the proof of our incentive structure we will analyse the possible malicious actions the actors can take during the two exchange stages of the mechanism.

\subsubsection{Malicious actions pickup exchange}

The object pickup exchange has the following possible malicious actions:
\begin{enumerate}
  \item Service Provider ($C$)
  \begin{enumerate}
    \item Signing $tx_1$
    \begin{enumerate}
      \item Wrong amount equivalent value ($Ev$)
      \item No signature from start address ($C_{sig}$)
      \item Wrong endpoint address ($B_{adr}$)
    \end{enumerate}
    \[\{Ev, C_{sig}, BC_{adr}^{2/2}\}\not\in tx_1\]
  \end{enumerate}
  \item Service Consumer ($A$)
  \begin{enumerate}
    \item Signing $tx_2$
    \begin{enumerate}
      \item Wrong amount transport reward ($Tr$)
      \item No signature from start address ($A_{sig}$)
      \item Wrong endpoint address ($BC_{adr}^{2/2}$)
    \end{enumerate}
    \[\{Tr, A_{sig}, BC_{adr}^{2/2}\}\not\in tx_2\]
    \item Ricardian Contract ($Rc$)
    \begin{enumerate}
      \item Giving wrong pickup exchange location ($A_{loc}$) in $Rc$
      \item Giving wrong dropoff exchange location ($B_{loc}$) in $Rc$
      \item Giving wrong dropoff exchange public key ($B_{pk}$) in $Rc$
      \item Change his mind on request
      \item Wrong change of ownership address
    \end{enumerate}
    \[\{A_{loc}, B_{loc}, B_{pk}, Ev, Tr\} \not\subseteq Rc\]
    \item Physical Object
    \begin{enumerate}
      \item Not giving the physical object to $C$
      \item Incorrect content
    \end{enumerate}
  \end{enumerate}
\end{enumerate}

\bigbreak
\noindent\textbf{Result malicious actions}
\bigbreak
\noindent\textbf{1.a.} $A$ can verify the content of the $tx_1$ before broadcasting, a transaction script byte-map can be found in Appendix D. $C_{sk}$ cannot be extracted from $tx_1$ after it has been signed and encoded into a transaction script. $A$ also has to wait till the transaction is confirmed into the ledger before proceeding with the process.
\[VerifyTransaction(Rc, tx_1)\colon \{Ev, B_{adr}\} \in Rc \land \{Ev, C_{sig},  BC_{adr}^{2/2}\} \in tx_1\]
\noindent\textbf{2.a.} C can verify the content of the $tx_2$
\[VerifyTransaction(Rc, tx_2)\colon\{Tr, B_{adr}\} \in Rc \land \{Tr, A_{sig}, BC_{adr}^{2/2}\} \in tx_2\]
\noindent\textbf{2.b.i.} C would lose the labor cost of moving to $A_{loc}$ \par
\noindent\textbf{2.b.ii.} C would lose the labor cost of moving to $A_{loc}$ \par
\noindent\textbf{2.b.iii.} C can verify the change of ownership of the $Rc$ before it is being broadcasted \par
\noindent\textbf{2.c.} $C$ can organise contract enforcement of $Rc$ through legal institutions because $Rc$ can be legally binding and $A_{loc}$ is known. \par

\subsubsection{Malicious actions drop-off exchange}

Possible malicious behaviour at object drop-off exchange:
\begin{enumerate}
  \item Service provider ($C$)
  \begin{enumerate}
    \item Ricardian Contract
    \begin{enumerate}
      \item Wrong change of ownership address
    \end{enumerate}
    \item Physical Object
    \begin{enumerate}
      \item Opening the package and keeping it
      \item Opening package and continue contract
    \end{enumerate}
  \end{enumerate}
  \item Endpoint actor ($B$)
  \begin{enumerate}
    \item Signing $tx_3$
    \begin{enumerate}
      \item Wrong amount ($\{tx_1, tx_2\}$)
      \item No signature from start address ($\{tx_1, tx_2\}^{1/2}$)
      \item Wrong endpoint address ($C_{adr}$)
    \end{enumerate}
    \[\{\{tx_1, tx_2\}^{1/2}, BC_{adr}^{2/2}, C_{adr}\}\not\in tx_3\]
    \item Ricardian Contract
    \begin{enumerate}
      \item Giving wrong drop-off exchange information in $Rc$
      \item State incorrect content
    \end{enumerate}
  \end{enumerate}
\end{enumerate}

\bigbreak
\noindent\textbf{Result malicious actions}
\bigbreak
\noindent\textbf{1.a.} $B$ can verify the change of ownership of the $Rc$ before it is being broadcasted \par
\noindent\textbf{1.b.i.} $C$ will lose $\{Ev, Tr\}$ \par
\noindent\textbf{1.b.ii.} Depending on content it might be prevented by chosen packaging, but will not be prevented through incentive mechanism. \par

\noindent\textbf{2.a.} $C$ can verify the content of the $tx_3$
\[VerifyTransaction(Rc, tx_3)\colon\{Ev, Tr, B_{adr}\} \in Rc \land \{\{tx_1, tx_2\}^{1/2}, BC_{adr}^{2/2}, C_{adr}\} \in tx_3\]
\noindent\textbf{2.b.i.} $C$ can keep package or return to $A$.\par
\noindent\textbf{2.b.ii.} If $B$ does not agree with content he will not sign to release the escrow. $C$ can keep package or return to $A$.\par

% Attack vectors:
% Eating the difference in equivalance cost if the reputation cost of the starting actor is benefitial to the actor eating the costs
