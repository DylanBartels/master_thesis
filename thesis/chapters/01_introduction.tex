The gig economy is in full effect, individual actors get paid for the execution of short term contracts and centralized companies intermediate in the supply and demand of this labour. With the intermediation of these parties the companies profit of margins and deny individuals full ownership of the value of the produced labour. Recent advancements in peer-to-peer technologies and decentralized posibilities interrest the academic domain if there are alternative options to shift towards decentralized solutions in the logistics domain.\par

\subsection{Initial Study}

% The relevancy of your research,
Recent successful technical innovations have been due to a shift from centralization to peer-to-peer services, examples of these are Uber, Airbnb and Kickstarter. In the domain of supply chain logistics this innovation has been lacking. O. Gallay et al. \cite{peer-to-peerDecentralizedLogistics} have proposed a peer-to-peer framework supporting interoperability between different actors in the logistics chains. This research lacks insight related to trust, network and technical implementation but establishes interest in the domain regarding implementation. \par
According to N. Hackius et al. \cite{hackius2017blockchain} surveys in the logistics domain show that there is a clear demand on what blockchain technology can realistically do for the domain. \par
With the recent progression in the domain of trustless value transference, research towards the applicability of this in the supply chain offers relevancy. In \cite{trustlessIntermediationInBCServiceMarket} M. Klems et al. have formulated possible implementation of trustless intermediation in blockchain based decentralized service marketplaces. The research topic arises if this or other intermediation solutions can also be applied to peer-to-peer logistics marketplaces and to which degree will there be a custodian in the process due to it including transferring a physical asset.

\subsection{Problem Statement}

The problem which will be explored in this study is how to trust actors to transport a physical good without trusting centralized intermediation and reputation systems are not a neccesity. Currently the transport domain operates around centralized reputation systems, whereby the companies with aggregated reputation and trust offer the service and carry responsibility for conflict resolution.\par
% Centralized companies offer many benefits and are consisidered to be more economically efficient than decentralized solutions\cite{}. The possible downsides of centralization are censorship, slow decision making and no room for competition.\par
However reputation loss might not be the only incentive available to achieve transport. In chapter 4 an alternative incentive construction will be demonstrated which aims to achieve decentralization, reduction on trusting reputation systems and every actor being able to fullfill every role in transport. The setup uses trustless escrow to lock the transport actor into not behaving hostile due to possible punishment. Deviation of rational behaviour would result in loss of value to counteract the inavalibility of current applied reputation loss punishment.

\subsubsection{Research Questions}

% The specific research questions to be answered in the project.
\bigbreak
\noindent We examine the proposed mechanism from the perspective of the following research question:
\begin{itemize}
  \item How can you provide a mechanism to facilitate peer-to-peer decentralized trustless transport contracts?
  % \item What are the specific problems and characteristics of a trustless decentralized peer-to-peer marketplace for transportation contracts?
\end{itemize}
\bigbreak
\noindent To contribute to a clear view of what the mechanism provides we state the following subquestions:
\bigbreak
% \noindent For the following subquestions marketplace is defined as a trustless decentralized peer-to-peer marketplace for transportation of goods.
\begin{itemize}
  \item Can trustless intermediation exists on this marketplace without centralized dispute resolution?
  \item How decentralized is the mechanism?
  \item To what degree do oracles play a role in the verification of the information?
  \item What level of anonymity is possible?
  \item Which attack vectors are possible to undermine this mechanism?
\end{itemize}

% How can you provide a mechanism to facilitate peer-to-peer decentralized trustless contracts for the transport of goods?

\subsubsection{Solution Outline}

Our solution uses a digital representation of the transport contract which includes the begin address, end address and the end address public key. This contracts will be called the asset and ownership on this contract will be tied to the owner of the physical asset which is intended to be transported with the process. When the transport actor is custodian of transporting the asset an equivalent value or more will be put in escrow which can be unlocked by the endpoint actor. The setup incentivezes for the transport actor to move the asset to the endpoint else losing the escrow equivalent value. \par
We chose to implement the escrow on the Bitcoin network due to it being well tested and offering a baseline environment in the decentralized domain.
% None specific cryptocurrency for escrow
% marketplace orderbook orders consist out of transport coordinates and asset being transported. Ownership of this translate to ownership in the physical domain.
% todo: figure

% \subsubsection{Why Blockchain?}
%
% Paper why blockchain
% cencorship resistance
% immutability
% ownership
% transparency
% downside
% upside

\subsubsection{Research Method}

The study will apply the action research methodology research method. Action research can be defined as an approach in which the action researcher and a client collaborate in the diagnosis of the problem and in the development of a solution based on the diagnosis. With this method a prototype of the marketplace and transport intermediation solution will be build in collaboration with Cargoledger. The methodology has the downside that biases might occur towards the chosen solution due to also being responsible for the development.\par

\subsection{Related Work}

\subsubsection{Building Trust in Decentralized Peer-to-Peer Electronic Communities \cite{buildTrust}}
\textbf{Summary}: Many players in electronic markets have to cope with much higher amount of uncertainty as to quality and reliability of the products they buy and the information they obtain from other peers in the respective online business communities. One way to address this uncertainty problem is to use information such as feedbacks about past experiences to help making recommendation and judgment on product quality and information reliability. This paper presents PeerTrust, a simple yet effective reputation-based trust mechanism for quantifying and comparing the trustworthiness of peers in a decentralized peer-to-peer electronic marketplace. There are three main contributions in this paper. First, we argue that the trust models based solely on feedbacks from other peers in the community is inaccurate and ineffective. We introduce three basic trust parameters in computing trust within an electronic community. In addition to feedbacks in terms of amount of satisfaction, we incorporate the feedback context such as the total number of transactions and the credibility of the feedback sources into the PeerTrust model for evaluating the trustworthiness of peers. Second, we develop a basic trust metric that combines the three critical parameters to compare and quantify the trustworthiness of peers. Third, we present a concrete method to validate the proposed trust model and report the set of initial experiments, showing the feasibility, costs, and benefits of our approach.\newline
\textbf{Differ from my approach}: Xiong \& Liu, 2003 focus on the wide domain of trust in decentralized peer-to-peer communities. With my specific case regarding a marketplace with transport of goods trust between the peers is an important aspect but is more contained in a specific domain than Xiong \& Liu.\newline
\textbf{Obtained result}: Xiong \& Liu, 2003 present PeerTrust a trust mechanism for building trust in peer-to-peer electronic communities. They identified three important trust parameters, these are: amount of satisfaction, number of interactions and balance factor of trust. They put the results into experiments which demonstrated the effectiveness, costs, and benefits of the approach.\newline
\textbf{Remaining open questions}: Looking at ways to make the approach more robust against malicious behaviors, such as collusions among peers. Combining trust management with intrusion detection to address concerns of sudden and malicious attacks. How to uniquely identify peers over time and associate their histories with them\newline

\subsubsection{The challenge of decentralized marketplaces \cite{challangeDecentralizedMarketplaces}}
\textbf{Summary}: Online trust systems are playing an important role in to-days world and face various challenges in building them. Billions of dollars of products and services are traded through elec- tronic commerce, files are shared among large peer-to-peer networks and smart contracts can potentially replace pa- per contracts with digital contracts. These systems rely on trust mechanisms in peer-to-peer networks like reputation systems or a trustless public ledger. In most cases, reputa- tion systems are build to determine the trustworthiness of users and to provide incentives for users to make a fair con- tribution to the peer-to-peer network. The main challenges are how to set up a good trust system, how to deal with se- curity issues and how to deal with strategic users trying to cheat on the system. The Sybil attack, the most important attack on reputation systems is discussed. At last match making in two sided markets and the strategy proofness of these markets are discussed.\newline
\textbf{Differ from my approach}: Very similair by giving a rundown of all the research done towards trust enforcements in p2p file sharing, decentralized markets and sybil attacks. \newline
\textbf{Obtained result}: B van Ijzendoorn gives a summary of academical research on decentralization, Sybil attacks, trust and peer-to-peer in relation to marketplaces. \newline
\textbf{Remaining open questions}: Not applied.\newline

\subsubsection{A Peer-To-Peer Platform for Decentralized Logistics \cite{peer-to-peerDecentralizedLogistics}}
\textbf{Summary}: We introduce a novel platform for decentralized logistics, the aim of which is to magnify and accelerate the impact o ered by the integration of the most recent advances in Information and Communication Technologies (ICTs) to multi-modal freight operations. The essence of our peer-to-peer (P2P) framework distributes the management of the logistics operations to the multiple actors according to their available computational resources. As a result, this new approach prevents the dominant players from capturing the market, ensures equal opportunities for di erent size actors, and avoids vendor lock-in. The latest ICTs such as In- dustrial Data Space (IDS), Blockchain, and Internet-of-Things (IoT) are used as basic building blocks which, together, enable the creation of a trusted and inte- grated platform to manage logistics operations in a fully decentralized way. While IDS technology allows for secured data exchange between the di erent parties in the logistics chain, Blockchain technology handles transaction history and agreements between parties in a decentralized way. IoT enables the gathering of real-time data over the logistics network, which can be securely exchanged between the di erent parties and used for managing the decision-making related to the control of the freight transportation activities. The practicability and the potential of the proposed platform is demonstrated with two use cases, involving various actors in the logistics chains.\newline
\textbf{Differ from my approach}: It is an academic research which originated from the logistics domain and aimed at solving the contradiction between interoperability and data sovereignty.\newline
\textbf{Obtained result}: High level decentralized logistics system architecture with data flows. Two initial use cases.\newline
\textbf{Remaining open questions}: Development of business models in parallel.\newline
