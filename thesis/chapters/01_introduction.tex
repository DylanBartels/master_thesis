The gig economy is in full effect, individual actors get payed for the execution of short term contracts and centralized companies intermediate in the supply and demand of this labour. With the intermediation of these parties the companies profit of margins and deny individuals full ownership of the value of the produced labour. Recent advancements in peer-to-peer technologies and decentralized posibilities interrest the academic domain if there are alternative options to shift towards decentralized solutions in the logistics domain.\par

\subsection{Initial Study}

% The relevancy of your research,
Recent successful technical innovations have been due to a shift from centralization to peer-to-peer services, examples of these are Uber, Airbnb and Kickstarter. In the domain of supply chain logistics this innovation has been lacking. O. Gallay et al. \cite{peer-to-peerDecentralizedLogistics} have proposed a peer-to-peer framework supporting interoperability between different actors in the logistics chains. This research lacks insight related to trust, network and technical implementation but establishes interest in the domain regarding implementation. \par
According to N. Hackius et al. \cite{hackius2017blockchain} surveys in the logistics domain show that there is a clear demand on what blockchain technology can realistically do for the domain. \par
With the recent progression in the domain of trustless value transference, research towards the applicability of this in the supply chain offers relevancy. In \cite{trustlessIntermediationInBCServiceMarket} M. Klems et al. have formulated possible implementation of trustless intermediation in blockchain based decentralized service marketplaces. The research topic arises if this or other intermediation solutions can also be applied to peer-to-peer logistics marketplaces and to which degree will there be a custodian in the process due to it including a physical process.

\subsection{Problem Statement}

The problem which will be explored in this study is the possiblity of creating a trustless transport system where reputation is not a neccesity. Currently the transport domain operates around centralized reputation systems, whereby the companies with aggregated reputation and trust offer the service and carry responsibility for conflict resolution.\par
% Centralized companies offer many benefits and are consisidered to be more economically efficient than decentralized solutions\cite{}. The possible downsides of centralization are censorship, slow decision making and no room for competition.\par
However reputation loss might not be the only incentive available to achieve transport. In chapter 4 an alternative incentive construction will be demonstrated which aims to achieve decentralization, reduction on trusting reputation systems and every actor being able to fullfill every role in transport. The setup uses trustless escrow to lock the transport actor into not behaving hostile due to possible punishment. Deviation of rational behaviour would result in loss of value to counteract the inavalibility of current applied reputation loss punishment.

\subsubsection{Research Questions}

% The specific research questions to be answered in the project.
\bigbreak
\noindent Main research question:
\begin{itemize}
  \item What are the specific problems and characteristics of a trustless decentralized peer-to-peer marketplace for transportation contracts?
\end{itemize}
\bigbreak
\noindent Subquestions:
\bigbreak
\noindent For the following subquestions marketplace is defined as a trustless decentralized peer-to-peer marketplace for transportation of goods.
\begin{itemize}
  \item Can trustless intermediation exists on this marketplace without a custodian for dispute prevention and resolution?
  %\item Which strategies can prevent sybil attacks on this marketplace?
  \item What level of anonymity is possible on this marketplace?
\end{itemize}

\subsubsection{Solution Outline}

None specific cryptocurrency for escrow
marketplace orderbook orders consist out of transport coordinates and asset being transported. Ownership of this translate to ownership in the physical domain.
todo: figure

\subsubsection{Why Blockchain?}

Paper why blockchain
cencorship resistance
immutability
ownership
transparency
downside
upside

\subsubsection{Research Method}

The study will apply the action research methodology research method. Action research can be defined as an approach in which the action researcher and a client collaborate in the diagnosis of the problem and in the development of a solution based on the diagnosis. With this method a prototype of the marketplace and transport intermediation solution will be build in collaboration with Cargoledger. The methodology has the downside that biases might occur towards the chosen solution due to also being responsible for the development.\par

\subsection{Related Work}
