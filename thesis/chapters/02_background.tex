\subsection{Bitcoin}

Introduced by the anonymous Satoshi Nakomoto in 2008, Bitcoin (BTC) is a decentralized peer-to-peer currency system. Bitcoin allows digital payments without going through a financial institution and solves the double spend problem by hashing timestamped transactions into an continues chain of hash-based Proof-of-Work (PoW) \cite{nakamoto2008bitcoin}. Payment are possible by creating transactions, signing them and sending them to Bitcoin addresses. The user has a public/private keypair whereby addresses are a mapping function of the public key and the private key can sign transactions. Creating of transactions is only possible if they have been send to one of the user owned addresses and are then called unspent transactions (UTXO).\par
The broadcasted transactions are send to the memory pool and included in blocks once the network has formed consensus on the correct order of transactions. To generate a block the miners  have to find a nonce value, peers than include it in a block which allows anybody to verify the PoW. Miners get rewarded upon generating a block thus incentivising to support the network with computational power. The generated computational power of the network, also expressed in hash rate or hash power, protects the integrity of the PoW chain. For a user the definition of owning a bitcoin is the right to sign a UTXO with your keypair.\par
The PoW mechanism solves the double spend problem by guaranteeing that the transaction is not spend twice when it is included in a block. For a malicious actor to double spend a BTC without detection they would have to recompute all previous blocks, so as long as the honest peers in the network exceed the malicious the integrity of the work is guaranteed \cite{nakamoto2008bitcoin}. \par
The average block time of the bitcoin network is 10 minutes, to guarantee that the double spend would not occur on average the receiver of the transaction would have to wait 10 minutes minus the time of last found block. Fast transactions (i.e. in the order of seconds) are not reliable because low cost attacks can be mounted effectively to spoof a transaction \cite{karame2012two}.

\subsubsection{Trustless}

Markus Klems et al. define trustless and trustless intermediation as follows:

\begin{displayquote}
   A system property which guarantees rules of interaction that are known to and agree upon by all participants of the system, and which cannot be unilaterally changed. These guarantees are enforced through, what we call trustless intermediation, a set of mechanisms for \textit{decentralizing the enforcement of rules in a system}, thereby removing the need for and existence of trusted intermediaries. \cite{trustlessIntermediationInBCServiceMarket}
\end{displayquote}

Bitcoin can be defined as a trustless system because once an UTXO is signed and broadcasted to the memory pool with a high enough transaction fee it is guaranteed that it will be put inside a block. No intermediation takes place for this mechanism to occur. The mechanism cannot be changed \textit{easily}, only if a hard forking is proposed including a code change. However this chain split would not count as bitcoin unless it is backed by the majority of hash rate thus being classified as longest chain \cite{nakamoto2008bitcoin}. \par
Extrapolating the definition to the logistics domain means that trustless logistics can be defined as an logistic contract which has predefined unchangeable rules without the need for trusted intermediaries.

\subsubsection{Script}

Bitcoin uses a stack-based scripting system for transactions which is intentionally not Turing-complete. A typical Bitcoin address is known as a Pay-to-PubKeyHash (P2PKH) address and can be identified by the $1$ prefix, an example of such an address alongside other address types can be found in Chapter 4.1. Table \ref{tab:addresses}\par
Another important address is the Pay-to-ScriptHash (P2SH) address identified by the $3$ prefix. The Pay-to-ScriptHash (P2SH) address type allows any of \textit{N} out of \textit{M} signatures to spend the UTXO available in the P2SH address. To send a P2SH transaction \textit{M} amount of public keys are needed and the required amount of signatures \textit{N} to allow to redeem from the address need to be given. \textit{M-of-N} multisignature P2SH addresses owned by $\{X_{pk}, Y_{pk}\}$ will be denoted as \(XY_{adr}^{m/2}\) from now on. \par
The Bitcoin scripting language provides operators which can be applied to created transaction scripts. The OP\_CHECKMULTISIG operators for multisig verification makes it possible to verify a generated transaction script for the content before it is broadcasted. \par

\subsection{Ricardian Contracts}

A Ricardian contract is designed to register a legally binding digital document to a specific object \cite{grigg2004ricardian}. The contract puts all information in a format which is parsable by software and humans. It represents a legal agreement between individuals and a protocol for integrating the agreement securely within a infrastructure. \par
The Ricardian contract can be used to form an agreement by forming the liability when trading with another party. It can represent a unit of goods or service and uses signed agreement between the parties which cannot be falsified once signed. \par
Similar to smart contracts \cite{buterin2014next} Ricardian Contracts can achieve taking out the middle man in contract construction, execution and enforcement. A Ricardian contract is acceptable within the existing legislation frameworks.

\subsection{Decentralized Marketplace}

Markus Klems et al. define centralized marketplaces\cite{trustlessIntermediationInBCServiceMarket} as providing mechanisms to facilitate efficient spot trades between numbers of sellers and buyers by providing match-making and payment transaction processes that are accompanied by trust-building mechanisms, most importantly, reputation and dispute resolution systems. Some examples of centralized marketplaces for logistics are Postmates for deliveries, Uber for the transport of people and Uship for shipping of goods. These peer-to-peer marketplaces facilitate the intermediation process of transport contracts.\par
Decentralized marketplaces facilitate the same spot trades without a central provider and intermediaries. The upside of such a mechanism is the lowering of the entry barrier \cite{einav2016peer}, no intermediation fees \cite{openbazaar}, increasing sensorship resistance \cite{decentralMarket}, and improving privacy \cite{decentralizedAnonymousReputation}.
% Oracles
