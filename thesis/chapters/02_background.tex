
\subsection{Blockchain}

Different definitions depending on the interpreter. Can be a ethics framework, speculative asset or a distributinon of rights to spend an unspend transaction (UTXO). There are no wrong answers.
Blockchain is the right to sign a UTXO (unspent transaction) with your keypair.

\subsubsection{Decentralized}

Centralization is very efficient compared to decentralization. The forming of consensus requers high amount of value by putting energy in computation with the proof of work algorithm, going back to the byzantine fault tolerance trusting other parties to comform to the input of information will be more costly. The function decentralization fulfills is raising the cost of attacking the consensus which is being formed every input stream of information captured in blocks. The sensorship resistance of decentralization is the function it excels at.\par
Following this premises in the domain of logistics centralizated entities fulfill trustworthy responsibllities which cannot compete with decentralized services. This is unless this service wants to benefit the cencorship resistance which could be made possible. Censoring is not often applied in logistics, some examples along many are:
\begin{enumerate}
  \item Transportation of written religious information in corresponding religion underpressed areas
  \item Transportation of illicit goods
  \item Entry level of competition
\end{enumerate}
The other option which logistics could benefit in the model of trade-offs is:
The costs of the controlling the goods to be transported and transporting being more constly than actually transporting the goods without intermediation of checking on the parameters of censorship.

\subsubsection{Trustless}

The definiton of trustless can be achieved if no entity is custodian of any process. In the domain of logistics a custodian will always be responsible for the actual transport of the physical good. This means that

\subsubsection{Multisigniture}



\subsection{Smart Contracts}

Smart contracts are digital representations of a contracts which will be activated once certain input activates parameters. The contracts promise to:
 enforce contracts automatically
Take out the middle man in contract construction, execution and enforcement
A normal contract
smart contracts are very difficult to implement well. They all trust on some oracle input which has to be correct for contracts to behave.  but how can you garantee the input is correct? Conesnsus on the correctness of data which is stored depends on the correctness of input. If the oracles who register the data are in full control of input they are lone ruler of correctness disregarding the consensus.

\subsection{Marketplace}

In the domain of logistical contracts for being the custodian for transport settled identities work on reputation based systems to create trust for transport. If decentralized trustless aggregation of contracts could take place in a orderbook everybody should be able to fulfill the same side of this marketplace, supply and demand.
The marketplace is a orderbook filled with digital representations of transport contracts. The contracts are demand of transport from place A to B, the supply is facilitators of this transport. Once the order is met and work has to be facilitated the supply picks up the asset and brings it to designated place of dropoff. The three actors: pickup, drop-off and transport are represented by a keypair which gives the right of ownership throughtout the process of the asset. This ownership is the right to the transport contract and asset being transported, the data it contains is the pickup and dropoff point.
